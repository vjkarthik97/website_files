\documentclass{article}
\usepackage[a4paper, top=2.25cm, bottom=2.5cm, left=2.25cm, right=2.25cm]%
{geometry}
\usepackage{amsfonts}   % if you want the fonts
\usepackage{amssymb}    % if you want extra symbols
\usepackage{setspace}
\usepackage{amsmath}
\usepackage{mathtools}
\usepackage{amssymb}
\usepackage[export]{adjustbox}
\setlength{\parindent}{0in}
\usepackage[utf8]{inputenc}
\usepackage[english]{babel}
\usepackage{hyperref}
\usepackage{graphicx}
\usepackage[dvipsnames]{xcolor}
\usepackage{hyperref}
% Package to place figures where you want them.
\usepackage{float}
\usepackage{array}

\newcommand{\htmlquad}{\space{0.5em}}

\usepackage{subfig,wrapfig}
\usepackage[
%backend=biber,
%style=alphabetic,
sorting=none
]{biblatex}
\bibliography{references}  

\begin{document}
\title{Integrated Control and Planning for Mobile Robots}
\author{Veejay Karthik J\\
Systems and Control Engineering\\
IIT Bombay}
\date{August 2023}


\maketitle

\section{Motivation}

Mobile Robots are a class of \textbf{underactuated} systems with inherent \textbf{non-holonomic constraints} that restrict their maneuverability. In practice, they are often deployed in space-constrained operating workspaces. Therefore, motion planning for such systems becomes a complex task, and it is significantly challenging when the exact obstacle configuration is not known beforehand. The mathematical models (unicycle, bicycle,etc) for mobile robots are typically driftless , and exhibit the following structure,
\begin{align}
\label{eqn:SystemDescription}
\centering
\dot{X} = g(X,U) \quad~;~\quad X \in Q \subseteq \mathbb{R}^n, U\in [U_m,U_M]\subset\mathbb{R}^m \quad~;~\quad m < n
\end{align}
Here, $g(X)\in \mathbb{R}^{n\times m}$ is typically a lipschitz function. The configuration space is given by $\mathcal{Q}$ and the set of obstacles is denoted by $\mathcal{O}_i$. From these definitions, the free configuration space $\mathcal{Q}_{\text{free}}$ is obtained as follows,
\begin{align}
    \label{eqn:FreeWorkspaces}
    \centering
    \mathcal{Q}_{\text{free}} = \{X\in\mathcal{Q}\;|\; X\cap\left(Q\cap (\cup_i \mathcal{O}_i)\right) = \emptyset \}
\end{align}

Suppose the initial conditions are given by $X_0$ at an initial time instant $t_0$ and the target goal is given by $X_f$ at a time instant $t=t_f$, a candidate motion plan $X_{\text{ref}}(t) \in Q$ is deemed admissible for navigation under the following conditions,
\begin{itemize}
    \item $X_{\text{ref}}(t)$ is a solution to (\ref{eqn:SystemDescription}), with $X_{\text{ref}}(t_0) = X_0$ and $X_{\text{ref}}(t_f) = X_f$.
    \item $X_{\text{ref}}(t) \in \mathcal{Q}_{\text{free}},\; \forall t\in[t_0,t_f]$  
\end{itemize}
\textbf{Under such conditions, the resulting motion plan $X_{\text{ref}}(t)$ is compliant for tracking through a careful design of a low-level tracking controller}. (The final time argument condition could be optionally relaxed if required). \textit{During operation in unknown environments, such amiable motion plans must be generated during run-time to accomplish the mission objectives based on the sensed information.}
\begin{figure}
    \centering
    \includegraphics[scale=0.5]{media/turtlebot3_burger_components.png}
    \caption{Caption}
    \label{fig:enter-label}
\end{figure}

\subsection{Integrated Planning and Control (IPC)}
IPC algorithms \textbf{seek motion plans as a sequence of feedback controllers over domains} in the navigable regions of the environment, rather than an explicit path/trajectory in the configuration space of the system. The IPC motion plans exhibit the following properties,
\begin{itemize}
    \item Each domain $\mathcal{D}_k$ is positively invariant under \textbf{feedback controller $\mathcal{F}_k$}, and is associated with a goal set $\mathcal{W}_k \subset \mathcal{D}_k$. Essentially, any initial state $X(t)\in \mathcal{D}_k$ evolves such that it converges onto $\mathcal{W}_k$ under the influence of $\mathcal{F}_k$ while remaining within $\mathcal{D}_k$ during the entire duration. Given a system as in (\ref{eqn:SystemDescription}), the following holds true,
    \begin{itemize}
        \item If $X(t=t_0) \in \mathcal{D}_k$, then $X(t=t_f) \in \mathcal{W}_k$ and $X(t)\in \mathcal{D}_k, \;\forall t \in [t_0,t_f]$
    \end{itemize}
    \item The domains $\mathcal{D}_k$ are sequenced in such a way that their goal sets $\mathcal{W}_k$ lies on \textbf{at least one other domain $\mathcal{D}_i$}, and each $\mathcal{D}_i \subset{Q}_{\text{free}}$. (This ensures overall connectivity between the domains $\mathcal{D}_k$)
    \begin{align*}
        \mathcal{W}_k \in \mathcal{D}_i, \quad i \neq k
    \end{align*}
    \begin{itemize}
        \item The initial configuration $\mathcal{S}$ in the navigation query is such that, $\mathcal{S}\in \bigcup \mathcal{D}_k$.
        \item The final configuration $\mathcal{T}$ in the navigation query is such that $\mathcal{T} \in \mathcal{W}_k$ for some $`k'$.
    \end{itemize}
\end{itemize}
Depending on the current state of the system, the $\mathcal{F}_k$ associated with the corresponding $\mathcal{D}_k$ is applied to the system. Suppose the domains $\mathcal{D}_k$ are sequenced in such a way that each $\mathcal{W}_k$ lies on a unique $\mathcal{D}_i, \; i\neq k$, the system's trajectories are guaranteed to safely converge onto $\mathcal{T}$ when the above-mentioned structures are established.
\newline 
\newline Mathematically, the geometric structure of $\mathcal{D}_k$ is a consequence of the underlying system dynamics for which $\mathcal{F}_k$ is designed for stability. \textbf{Essentially, $\mathcal{D}_k$ can be thought of as a geometric manifestation of the feedback controller $\mathcal{F}_i$. Therefore, given the existence of a feedback controller, the problem of mobile robot navigation effectively becomes a problem of computational geometry (involving the sequencing of $\mathcal{D}_k$) which can often be solved much faster than optimization/optimal control problems.} This can potentially suit real-time implementations where online decision-making in the face of uncertainties is vital during operation in unknown/dynamic environments.
\printbibliography
\end{document}
